\documentclass[ucs,9pt]{beamer}

% Copyright 2004 by Till Tantau <tantau@users.sourceforge.net>.
%
% In principle, this file can be redistributed and/or modified under
% the terms of the GNU Public License, version 2.
%
% However, this file is supposed to be a template to be modified
% for your own needs. For this reason, if you use this file as a
% template and not specifically distribute it as part of a another
% package/program, I grant the extra permission to freely copy and
% modify this file as you see fit and even to delete this copyright
% notice.
%
% Modified by Tobias G. Pfeiffer <tobias.pfeiffer@math.fu-berlin.de>
% to show usage of some features specific to the FU Berlin template.

% remove this line and the "ucs" option to the documentclass when your editor is not utf8-capable
\usepackage[utf8x]{inputenc}    % to make utf-8 input possible
\usepackage[english]{babel}     % hyphenation etc., alternatively use 'german' as parameter

\include{fu-beamer-template}  % THIS is the line that includes the FU template!

\usepackage{arev,t1enc} % looks nicer than the standard sans-serif font
% if you experience problems, comment out the line above and change
% the documentclass option "9pt" to "10pt"

% image to be shown on the title page (without file extension, should be pdf or png)
\titleimage{fu_500}

\title[Short Paper Title] % (optional, use only with long paper titles)
{Smartwatches and Fitness trackers: Cyberphysical Privacy and Security Threats}

\subtitle
{IoT and Security}

\author[Author, Another] % (optional, use only with lots of authors)
{Henrik Strangalies} % F.~Author \and S.~Another
% - Give the names in the same order as the appear in the paper.

\institute[FU Berlin] % (optional, but mostly needed)
{Freie Universität Berlin}
% - Keep it simple, no one is interested in your street address.

%\date[CFP 2003] % (optional, should be abbreviation of conference name)
%{Conference on Fabulous Presentations, 2003}
% - Either use conference name or its abbreviation.
% - Not really informative to the audience, more for people (including
%   yourself) who are reading the slides online

\subject{Technical Computer Science}
% This is only inserted into the PDF information catalog. Can be left
% out.

% you can redefine the text shown in the footline. use a combination of
% \insertshortauthor, \insertshortinstitute, \insertshorttitle, \insertshortdate, ...
\renewcommand{\footlinetext}{\insertshortinstitute, \insertshorttitle, \insertshortdate}

% Delete this, if you do not want the table of contents to pop up at
% the beginning of each subsection:
\AtBeginSubsection[]
{
  \begin{frame}<beamer>{Outline}
    \tableofcontents[currentsection,currentsubsection]
  \end{frame}
}

\begin{document}

\begin{frame}[plain]
  \titlepage
\end{frame}

\begin{frame}{Outline}
  \tableofcontents
  % You might wish to add the option [pausesections]
\end{frame}

\section{Motivation}



\begin{frame}{Motivation behind IoT wearables: Smartwatches and Fitness Trackers}
  % - A title should summarize the slide in an understandable fashion. 
  %   for anyone how does not follow everything on the slide itself.
  \begin{itemize}
  \item Wearable devices have become increasingly popular
   due to their convenience and functionality.
  \item Enabling users to perform various tasks such as \textbf{making payments}, \textbf{monitoring health}, and \textbf{receiving messages}.
  
  \pause 
  
  \item  Along with these benefits, wearables bring forth security and privacy concerns:
  \begin{itemize}
	\item \textbf{Data Collection.}
	\item \textbf{Data Transfer} between wearable device and phone.
	\item Applications of \textbf{third-party companies}.
	\item Location-based threats.
  \end{itemize}
  \end{itemize}
\end{frame}


\section{Panorama of Security \& Privacy Considerations with IoT wearables}

\begin{frame}[fragile]{Panorama of Security \& Privacy Considerations with IoT wearables}

\includegraphics[width=1\linewidth]{imgs/ASurveyofWearableDevicesandChallenges}

\end{frame}


\subsection{Threats to Confidentiality}

\begin{frame}{Confidentiality}
	\begin{alertblock}{Definition}
	Threats to Confidentiality encompasses those where attackers get unauthorised  access to information using techniques such as eavesdropping  the wireless channel.	
	\end{alertblock}
	
	% 	Most existing wearable devices use Bluetooth Low Energy (BLE) as the major means of communication.
	
	
\end{frame}

\begin{frame}{Eavesdropping}
	\begin{itemize}
		\item Eavesdropping is the unauthorized real-time interception of a private communication which can expose user’s personal information to an attacker.
%Particularly, wearable devices using BLE communication protocol can suffer from eavesdropping.	
		\item The authors of the Open Effect Report from 2016 \cite{b7} investigated BLE privacy provision in number of fitness tracking devices such as Fitbit Charge HR, Jawbone UP 2, Garmin Vivosmart, 	Apple Watch, and Xiaomi Mi Band and came to the conclusion all tested devices, except Apple Watch, use the static device addresses that allowed attackers to \textbf{track user information such as location, time of fitness activities, and reversing user profile} by eavesdropping on these devices’ communications.
	\end{itemize}
\end{frame}

\begin{frame}{Traffic analysis}
	\begin{itemize}
		\item Traffic analysis attacks in the context of wearables involve monitoring communication patterns between devices.
		\item Privacy vulnerabilities have been identified in Bluetooth Low Energy (BLE) communication between fitness trackers and smartphones. 
		\item Adversaries can track users by analyzing BLE advertisements and static device addresses. 
		\item User activities can be inferred from the size and number of data packets in BLE traffic, even if the packets are encrypted. 
		\item Unique walking patterns can also be used to identify individuals within a small group, even with random 	addresses [1].
		\item It has been shown that the majority of fitness 		trackers use unchanged BLE addresses during advertising, 		making it feasible to track them. 
		\item The BLE traffic of the 		fitness trackers is found to be correlated with the intensity 		of the user’s activity, enabling an eavesdropper to determine 		the \textbf{user’s current activity} (walking, sitting, idle, or running) 		through analysis of the BLE traffic. 
	\end{itemize}
\end{frame}

\begin{frame}{Infomration Gathering Attacks.}
	\begin{itemize}
		\item Passive monitoring of wearable device transmissions enables adversaries to collect data exchanged between wearables and their hubs. 
		\item This information can be used for information gathering attacks, including breaking key exchanges in Bluetooth Low Energy (BLE) pairing and gathering information about user's other devices. 
		\item Researchers have demonstrated attacks that break BLE legacy pairing, infer keystrokes on smartphone touchpads using smartwatch motion sensors, decode keystrokes on keyboards using smartwatch sensors, and infer a user's personal PIN sequence using wearable devices.
		\item Adversaries can gain access to smartwatches by installing malicious applications to record sensor activities. 
		\item These attacks leverage sensor data captured by wearables and can be executed by sniffing BLE communications or installing malicious apps on wearables \cite{b1}.
	\end{itemize}
\end{frame}

\subsection{Integrity}


\begin{frame}{Threats to Integrity}
		\begin{alertblock}{Definiton}
			Threats to Integrity includes the cases  where attackers alter data or information without authorisation.  Threats to Availability are the situations where attackers act  to deny services to the entities who are authorised to use them.
		\end{alertblock}
	
		%Integrity is a crucial security requirement for wearable systems, particularly due to the sensitivity and privacy of the collected data. 	
		%Ensuring that data remains unaltered during transmission and reaches only authorized parties is paramount.	 
		%Various studies in the literature have evaluated the integrity of 		wearable device systems, identifying vulnerabilities in three attack categories: Modification Attacks, Replay Attacks, and Masquerade Attacks [1]. Additionally, this section comprises an overview of possible data breaches.
\end{frame}

¸TODO: Doch lieber nur eine Folie für alle attacken nehmen...

\begin{frame}{Make Titles Informative.}
\end{frame}

\begin{frame}{Make Titles Informative.}
\end{frame}



\subsection{Availability}

\begin{frame}

\begin{alertblock}{Definiton}
 Threats to Availability are the situations where attackers act  to deny services to the entities who are authorised to use them.
\end{alertblock}
\end{frame}

\section{Threats to security and privacy from accelerometer data}



\section{Inferring Typed Words}



\section*{Summary}

\begin{frame}{Take Home Messages}

  % Keep the summary *very short*.
  \begin{itemize}
  \item
    The \alert{first main message} of your talk in one or two lines.
  \item
    The \alert{second main message} of your talk in one or two lines.
  \item
    Perhaps a \alert{third message}, but not more than that.
  \end{itemize}
  
  % The following outlook is optional.
  \vskip0pt plus.5fill
  \begin{itemize}
  \item
    Outlook
    \begin{itemize}
    \item
      Something you haven't solved.
    \item
      Something else you haven't solved.
    \end{itemize}
  \end{itemize}
\end{frame}



% All of the following is optional and typically not needed. 
\appendix
\section<presentation>*{\appendixname}
\subsection<presentation>*{For Further Reading}

\begin{frame}[allowframebreaks]
  \frametitle<presentation>{For Further Reading}
    
  \begin{thebibliography}{10}
    
  \beamertemplatebookbibitems
  % Start with overview books.

  \bibitem{Author1990}
    A.~Author.
    \newblock {\em Handbook of Everything}.
    \newblock Some Press, 1990.
 
    
  \beamertemplatearticlebibitems
  % Followed by interesting articles. Keep the list short. 

  \bibitem{Someone2000}
    S.~Someone.
    \newblock On this and that.
    \newblock {\em Journal of This and That}, 2(1):50--100,
    2000.
  \end{thebibliography}
\end{frame}

\end{document}
